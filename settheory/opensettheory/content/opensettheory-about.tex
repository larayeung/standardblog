\chapter*{About}
This book is an Open Education Resource. It is written for students with a little background in logic, and some high school mathematics. It aims to scratch the tip of the surface of the philosophy of set theory. By the end of this book, students reading it might have a sense of:
\begin{enumerate}
	\item why set theory came about; 
	\item how to reduce large swathes of mathematics to set theory + arithmetic;
	\item how to embed arithmetic in set theory;
	\item what the cumulative iterative conception of set amounts to;
	\item how one might try to justify the axioms of ZFC.
\end{enumerate}
The book grew out of a short course that I (Tim Button) teach at Cambridge. Before me, it was lectured by Luca Incurvati and Michael Potter. In writing this book---and the course, more generally---I was hugely indebted to both Luca and Michael. I want to offer both of them my heartfelt thanks.

\
\\
The book also draws material from the \textit{Open Logic Text}. In particular, chapters \ref{ch:Naive}--\ref{ch:Functions} are drawn (with only tiny changes) from the \emph{Open Logic Text}. Reciprocally, I (Tim) am contributing the material I have written back to the Open Logic Project. Please see 
\href{http://openlogicproject.org/}{openlogicproject.org} for
more information. 

\
\\This book was released \today, under a Creative Commons license (CC BY 4.0; full details are available at the CC \href{http://creativecommons.org/licenses/by/4.0/}{website}). The following is a human-readable summary of (and not a substitute for) that license:

\
\\
{\footnotesize \textbf{License details.} You are free to:\\
	--- \emph{Share}: copy and redistribute the material in any medium or format\\
	--- \emph{Adapt}: remix, transform, and build upon the material
	for any purpose, even commercially.
	\\Under the following terms:\\
	--- \emph{Attribution}: You must give appropriate credit, provide a link to the license, and indicate if changes were made. You may do so in any reasonable manner, but not in any way that suggests the licensor endorses you or your use.}