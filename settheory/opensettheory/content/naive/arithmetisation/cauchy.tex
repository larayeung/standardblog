% Part: sets-functions-relations
% Chapter: arithmetisation
% Section: Appendix: Cauchy
%
\documentclass[../../../include/open-logic-section]{subfiles}


\begin{document}
	
	\olfileid{sfr}{arith}{cauchy}
	
	\olsection{Appendix: the reals as Cauchy sequences}\ollabel{cauchy}
In this chapter, we constructed the reals as Dedekind cuts. In this appendix, I want to explain an alternative construction. It builds on Cauchy's definition of (what we now call) a Cauchy sequence; but the use of this definition to \emph{construct} the reals is due to other nineteenth-century authors, notably Weierstrass, Heine, M\'{e}ray and Cantor. (For a nice history, see \citeauthor{OConnorRobertson:RN} \citeyear{OConnorRobertson:RN}.)

Before we get to the nineteenth century, it's worth considering Simon Stevin (1548--1620). In brief, Stevin realised that we can think of each real in terms of its decimal expansion. Thus even an irrational number, like $\sqrt{2}$, has a nice decimal expansion, beginning:
	$$1.41421356237\ldots$$
It is very easy to model decimal expansions in set theory: simply consider them as functions $d \colon \Nat \to \Nat$, where $d(n)$ is the $n$th decimal place that we are interested in. We will then need a bit of tweak, to handle the bit of the real number that comes before the decimal point (here, just $1$). We will also need a further tweak (an equivalence relation) to guarantee that, for example, $0.999\ldots = 1$. But it is not difficult to offer a perfectly rigorous construction of the real numbers, in the manner of Stevin, within set theory. 

Stevin is not our focus. (For more on Stevin, see \citealt{KatzKatz2012}.) But here is a closely related thought. Instead of treating $\sqrt{2}$'s decimal expansion directly, we can instead consider a  \emph{sequence} of increasingly accurate rational approximations to $\sqrt{2}$, by considering the increasingly precise expansions: 
	$$1, 1.4, 1.414, 1.4142, 1.41421,\ldots$$
The idea that reals can be considered via ``increasingly good approximations'' provides us with the basis for another sequence of insights (akin to the realisations that we used when constructing $\Rat$ from $\Int$, or $\Int$ from $\Nat$). The basic insights are these:
	\begin{enumerate}
		\item Every real can be written as a (perhaps infinite) decimal expansion.
		\item The information encoded by a (perhaps infinite) decimal expansion can be equally be encoded by a sequence of rational numbers.
		\item A sequence of rational numbers can be thought of as a function from $\Nat$ to $\Rat$; just let $f(n)$ be the $n$th rational in the sequence.
	\end{enumerate}
Of course, not just \emph{any} function from $\Nat$ to $\Rat$ will give us a real number. For instance, consider this function:
	\begin{align*}
		f(n) &=\begin{cases}
			1 & \text{if }n\text{ is odd}\\
			0 &\text{if }n\text{ is even}
		\end{cases}
	\end{align*}
Essentially the worry here is that the sequence $0,1,0,1,0,1,0,\ldots$ doesn't seem to ``hone in'' on any real. So: to ensure that we consider sequences which do hone in on some real, we need to restrict our attention to sequences which have some \emph{limit}. 

We have already encountered the idea of a limit, in \olref[sfr][history][limits]{limits}. But we cannot use \emph{quite} the same definition as we used there. The expression ``$(\forall \epsilon>0)$'' there tacitly involved quantification over the real numbers; and we were considering the limits of functions on the real numbers; so invoking that definition would be to help ourselves to the real numbers; and they are exactly what we were aiming to \emph{construct}. Fortunately, we can work with a closely related idea of a limit. 
	\begin{defn}\label{def:CauchySequence}
		A function $f: \Nat \to \Rat$ is a \emph{Cauchy sequence} iff for any positive $\epsilon \in \Rat$ we have that $(\exists \ell \in \Nat)(\forall m, n > \ell)|f(m) - f(n)| < \epsilon$. 
	\end{defn}\noindent	
The general idea of a limit is the same as before: if you want a certain level of precision (measured by $\epsilon$), I can give you a ``region'' to look in (any input greater than $\ell$). And it is easy to see that our sequence $1$, $1.4$, $1.414$, $1.4142$, $1.41421$\ldots has a limit: if you want to approximate $\sqrt{2}$ to within an error of $\nicefrac{1}{10^{n}}$, then just look to any entry after the $n$th.

The obvious thought, then, would be to say that a real number just \emph{is} any Cauchy sequence. But, as in the constructions of $\Int$ and $\Rat$, this would be too na\"{i}ve: for any given real number, multiple different Cauchy sequences indicate that real number. A simple way to see this as follows. Given a Cauchy sequence $f$, define $g$ to be exactly the same function as $f$, except that $g(0)\neq f(0)$. Since the two sequences agree everywhere after the first number, we will (ultimately) want to say that they have the same limit, in the sense employed in Definition \ref{def:CauchySequence}, and so should be thought of ``defining'' the same real. So, we should really think of these Cauchy sequences as the same real number.

Consequently, we again need to define an equivalence relation on the Cauchy sequences, and identify real numbers with equivalence relations. First we need the idea of a function which tends to $0$ in the limit. For any function $h : \Nat \to \Rat$, say that \emph{$h$ tends to $0$} iff for any positive $\epsilon \in \Rat$ we have that $(\exists \ell \in \Nat)(\forall n > \ell)|f(n)| < \epsilon$.\footnote{Compare this with the definition of $\lim_{x \mathord{\rightarrow}\infty}f(x) = 0$ in \olref[sfr][history][limits]{limits}.} Further, where $f$ and $g$ are functions $\Nat \to \Rat$, let $(f-g)(n) = f(n) - g(n)$. Now define:
	\begin{align*}
		f \Realequiv g& \text{ iff }(f-g)\text{ tends to }0
	\end{align*}
We need to check that $\Realequiv$ is an equivalence relation; and it is. We can then, if we like, define the reals as the equivalence classes, under $\Realequiv$, of all Cauchy sequences from $\Nat \to \Rat$.
\begin{prob}
	Let $f(n) = 0$ for every $n$. Let $g(n) = \frac{1}{(n+1)^2}$. Show that both are Cauchy sequences, and indeed that the limit of both functions is $0$, so that also $f \sim_\Real g$. 
\end{prob}

Having done this, we shall as usual write $\equivrep{f}{\Realequiv}$ for the equivalence class with $f$ as !!a{element}. However, to keep things readable, in what follows I will drop the subscript and write just $\equivrep{f}{}$. We also stipulate that, for each $q \in \Rat$, we have $q_{\Real} = \equivrep{c_{q}}{}$, where $c_{q}$ is the constant function $c_q(n) = q$ for all $n \in \Nat$. We then define basic relations and operations on the reals, e.g.:
	\begin{align*}
		\equivrep{f}{} + \equivrep{g}{} &= 	\equivrep{(f + g)}{} \\
		\equivrep{f}{} \times 	\equivrep{g}{} &= \equivrep{(f \times g)}{} 
	\end{align*}
where $(f + g)(n) = f(n) + g(n)$ and $(f \times g)(n) = f(n) \times g(n)$. Of course, we also need to check that each of  $(f + g)$, $(f-g)$ and $(f\times g)$ are Cauchy sequences when $f$ and $g$ are; but they are, and I leave this to you.

Finally, we define we a notion of order. Say $\equivrep{f}{}$ is \emph{positive} iff both $\equivrep{f}{}\neq 0_\Rat$ and $(\exists \ell \in \Nat)(\forall n > \ell)0 < f(n)$. Then say $\equivrep{f}{} < \equivrep{g}{}$ iff $\equivrep{(g - f)}{}$ is positive. We have to check that this is well-defined (i.e.\ that it does not depend upon choice of ``representative'' function from the equivalence class). 
%\begin{align*}
%	\equivrep{f}{} \leq \equivrep{g}{} \text{ iff }&\equivrep{f}{} = \equivrep{g}{} \text{ or }\\
%	&(\exists \ell \in \Nat)(\forall n \in \Nat)(\ell < n \lonlyif f(n) \leq g(n))
%\end{align*}
But having done this, it is quite easy to show that these yield the right algebraic properties; that is:
\begin{thm}\ollabel{thm:cauchyorderedfield}
	The Cauchy sequences constitute an \emph{ordered field}.
\end{thm}\noindent
I leave this an exercise. 
\begin{prob}
	Prove that the Cauchy sequences constitute an ordered field.
\end{prob}
It is harder to prove that the reals, so constructed, have the Completeness Property, so I will prove that for you:
\begin{thm}
	Every non-empty set of Cauchy sequences with an upper bound has a least upper bound.
\end{thm}
\begin{proof}[Proof sketch]
Let $S$ be any non-empty set of Cauchy sequences with an upper bound. So there is some $p \in \Rat$ such that $p_{\Real}$ is an upper bound for $S$. Let $r \in S$; then there is some $q \in \Rat$ such that $q_{\Real} < r$. So if a least upper bound on $S$ exists, it is between $q_\Real$ and $p_\Real$ (inclusive). 

We will hone in on the l.u.b., by approaching it simultaneously from below and above. In particular, we define two functions, $f, g \colon \Nat \to \Rat$, with the aim that $f$ will hone in on the l.u.b.\ from above, and $g$ will hone on in it from below. We start by defining:
\begin{align*}
	f(0) &= p \\
	g(0) &= q
\end{align*}
Then, where $a_n = \frac{f(n) + g(n)}{2}$, let:\footnote{This is a recursive definition. But we have not \emph{yet} given any reason to think that recursive definitions are ok.}
\begin{align*}
	f(n+1) &=
	\begin{cases}
		a_n &\text{if $(a_n)_\Real$ is an upper bound for }S\\
		f(n)&\text{otherwise}
	\end{cases}\\
	g(n+1) &=
	\begin{cases}
		a_n &\text{if $(a_n)_\Real$ is a lower bound for }S\\
	 	g(n) &\text{otherwise}
	\end{cases}
\end{align*}
Both $f$ and $g$ are Cauchy sequences. (This can be checked fairly easily; but I leave it as an exercise.) Note that the function $(f-g)$ tends to $0$, since the difference between $f$ and $g$ halves at every step. Hence $\equivrep{f}{} = \equivrep{g}{}$. 

To show that  $\equivrep{f}{}$ is an upper bound for $S$, I will invoke \olref{thm:cauchyorderedfield}. Let $\equivrep{h}{} \in S$ and suppose, for reductio, that $\equivrep{f}{} < \equivrep{h}{}$, so that $0_\Real < \equivrep{(h-f)}{}$. Since $f$ is a monotonically decreasing Cauchy sequence, there is some $k \in \Nat$ such that $\equivrep{(c_{f(k)} - f)}{} < \equivrep{(h-f)}{}$. So:
$$(f(k))_\Real  = \equivrep{c_{f(k)}}{} < \equivrep{f}{} + \equivrep{(h-f)}{} = \equivrep{h}{}$$
contradicting the fact that $(f(k)_\Real)$ is, by construction, an upper bound for $S$.

In an exactly similar way, we can show that $\equivrep{g}{}$ is a lower bound for $S$. So $\equivrep{f}{} = \equivrep{g}{}$ is the \emph{least} upper bound for $S$.
\end{proof}


\end{document}