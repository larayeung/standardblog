% Part:sets-functions-relations
% Chapter: sets
% Section: equinumerous-sets

\documentclass[../../../include/open-logic-section]{subfiles}

\begin{document}

\olfileid{sfr}{set}{equ}

\olsection{Equinumerosity}\ollabel{equ}
We have an intuitive notion of ``size'' of sets, which works fine for
finite sets. But what about infinite sets? If we want to come up with
a formal way of comparing the sizes of two sets of \emph{any} size, it
is a good idea to start by defining when sets are the same size. Here is Frege:
\begin{quote}
	If a waiter wants to be sure that he has laid exactly as many knives as plates on the table, he does not need to count either of them, if he simply lays a knife to the right of each plate, so that every knife on the table lies to the right of some plate. The plates and knives are thus uniquely correlated to each other, and indeed through that same spatial relationship. \citep[\S70]{Frege1884}
\end{quote}
The insight of this passage can be brought out through a formal definition:
\begin{defn}\ollabel{comparisondef}$A$ is \emph{equinumerous} with $B$, written $\cardeq{A}{B}$, iff there is !!a{bijection} $f \colon A \to B$. 
\end{defn}
\begin{prop}\ollabel{equinumerosityisequi}
Equinumerosity is an equivalence relation.
\end{prop}
\begin{proof} We must show that equinumerosity is reflexive, symmetric, and transitive. Let $A, B$, and $C$ be sets.

\emph{Reflexivity.} The identity map $1_A \colon A \to A$, where
  $1_A (x) = x$ for all $x \in A$, is !!a{bijection}. So $\cardeq{A}{A}$.

\emph{Symmetry.} Suppose $\cardeq{A}{B}$, i.e.\ there
  is !!a{bijection} $f\colon A \to B$. Since $f$ is
  !!{bijective}, its inverse $f^{-1}$ exists and is also
  !!{bijective}. Hence, $f^{-1}\colon B \to A$ is a !!{bijection}, so $\cardeq{B}{A}$.

\emph{Transitivity.} Suppose that $\cardeq{A}{B}$ and $\cardeq{B}{C}$, i.e.\ there are !!{bijection}s $f\colon A \to B$ and $g\colon B \to
  C$. Then the composition $\comp{f}{g}\colon A \to C$ is
  !!{bijective}, so that $\cardeq{A}{C}$.
\end{proof}
\begin{prop}
If $\cardeq{X}{Y}$, then $X$ is !!{enumerable} if
and only if $Y$ is.
\end{prop}
\begin{proof}
Suppose $\cardeq{X}{Y}$, so there is some !!{bijection} $g \colon X \to Y$. Without loss of generality, suppose that $X$ is
!!{enumerable}. Then either $X  = \emptyset$ or there is
!!a{bijection} whose range is $X$ and whose domain is either $\Nat$ or an initial sequence of natural numbers. If $X =
\emptyset$, then $Y = \emptyset$ also (otherwise there would be
some~$y \in Y$ with no $x \in X$ such that $g(x) = y$).  So suppose we have our !!{bijection} $g$. Then $\comp{f}{g}$ is !!a{bijection}, so that $Y$ is !!{enumerable}.
\end{proof}

\begin{prob}
Show that if $\cardeq{X}{U}$ and and $\cardeq{Y}{V}$, and $X \cap Y = U \cap V = \emptyset$,
then $\cardeq{X \cup Y}{U \cup V}$.
\end{prob}

\begin{prob}
Show that if $X$ is infinite and !!{enumerable}, then $\cardeq{X}{\Nat}$.
\end{prob}

\end{document}
