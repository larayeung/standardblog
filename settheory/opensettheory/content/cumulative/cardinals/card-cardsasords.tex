\documentclass[../../../include/open-logic-section]{subfiles}

\begin{document}
	\olfileid{sfr}{cardinals}{card-cardsasords}
	\olsection{Cardinals as ordinals}\ollabel{card-cardsasords}
In fact, our theory of cardinals will just make (shameless) use of our theory of ordinals. That is: we will just define cardinals as certain specific ordinals. In particular, we will offer the following:
\begin{defn}\ollabel{defcardinalasordinal}
	If $A$ can be well-ordered, then $\cardof{A}$ is the least ordinal $\gamma$ such that $\cardeq{A}{\gamma}$. For any ordinal $\gamma$, we say that $\gamma$ is a \emph{cardinal} iff $\gamma = \cardof{\gamma}$. 
\end{defn}\noindent
I just used the phrase ``$A$ can be well-ordered''. As is almost always the case in mathematics, the modal-locution here is just a hand-waving gloss on an existential claim: to say ``$A$ can be well-ordered'' is just to say ``there is a relation which well-orders $A$''. 

But there is a snag with \olref{defcardinalasordinal}. We would like it to be the case that \emph{every} set has a size, i.e.\ that $\cardof{A}$ exists for every $A$. The definition I just gave, though, begins with a conditional: ``\emph{If} $A$ can be well-ordered\ldots''. If there is some set $A$ which cannot be well-ordered, then our definition will simply fail to define an object $\cardof{A}$.

So, to use \olref{defcardinalasordinal}, we need a guarantee that every set can be well-ordered. Sadly, though, this guarantee is unavailable in $\ZF$. So, if we want to use \olref{defcardinalasordinal}, there is no alternative but to add a new axiom, such as:
\begin{axiom*}[Well-Ordering]
	Every set can be well-ordered.
\end{axiom*}\noindent
We will discuss whether the Well-Ordering Axiom is acceptable in chapter \ref{ch:choice}. From now on, though, we will simply help ourselves to it. And, using it, it is quite straightforward to prove that cardinals (as defined in \olref{defcardinalasordinal}) exist and behave nicely:
\begin{lem}\ollabel{lem:CardinalsExist}For every set $A$:
	\begin{enumerate}
		\item\ollabel{cardofaexists} $\cardof{A}$ exists and is unique
		\item\ollabel{cardofaapprox}  $\cardeq{\cardof{A}}{A}$
		\item\ollabel{cardofaidem}  $\cardof{A}$ is a cardinal, i.e.\ $\cardof{A} = \cardof{\cardof{A}}$
	\end{enumerate}
\end{lem}
\begin{proof}
	Fix $A$. By Well-Ordering, there is a well-ordering $\tuple{A, R}$. By \olref[sfr][ordinals][ordtype]{thmOrdinalRepresentation}, $\tuple{A, R}$ is isomorphic to a unique ordinal, $\beta$. So  $\cardeq{A}{\beta}$. By Transfinite Induction, there is a uniquely least ordinal, $\gamma$, such that $\cardeq{A}{\gamma}$. So $\cardof{A} = \gamma$, establishing \olref{cardofaexists} and \olref{cardofaapprox}. To establish \olref{cardofaidem}, note that if $\delta \in \gamma$ then $\cardless{\delta}{A}$, by our choice of $\gamma$, so that also $\cardless{\delta}{\gamma}$ since equinumerosity is an equivalence relation (\olref[sfr][set][equ]{equinumerosityisequi}). So $\gamma = \cardof{\gamma}$. %So, for reductio, suppose that there is some ordinal $\delta \in \gamma$ such that $\cardeq{\gamma}{\delta}$. Then, $\cardeq{\cardeq{A}{\gamma}}{\delta}$ so that $\cardeq{A}{\delta}$ by \olref[sfr][set][equ]{equinumerosityisequi}, which contradicts the choice of $\gamma$ as the least ordinal such that $\cardeq{A}{\gamma}$. 
\end{proof}\noindent
This next result guarantees Cantor's Principle, and more besides. (Note that cardinals inherit their ordering from the ordinals, i.e.\ $\cardfont{a} < \cardfont{b}$ iff $\cardfont{a} \in \cardfont{b}$. In formulating this, I will use Fraktur letters for objects we know to be cardinals. This is fairly standard. A common alternative is to use Greek letters, since cardinals are ordinals, but to choose them from the middle of the alphabet, e.g.: $\kappa, \lambda$.):
\begin{lem}\ollabel{lem:CardinalsBehaveRight}For any sets $A$ and $B$:
		\begin{align*}
			\cardeq{A}{B} &\text{ iff }\cardof{A} = \cardof{B}\\
			\cardle{A}{B} &\text{ iff }\cardof{A} \leq \cardof{B}\\
			\cardless{A}{B}&\text{ iff }\cardof{A} < \cardof{B}
		\end{align*}
\end{lem}
	\begin{proof}
	I will prove the left-to-right direction of the second claim (the other cases are similar, and left as an exercise). So, consider the following diagram:
	\begin{center}
		\begin{tikzpicture}
		\node (nodea) {$A$};
		\node[right = 6em of nodea] (nodeb) {$B$};
		\node[below = 2em of nodea] (nodecarda) {$\cardof{A}$};
		\node[below = 2em of nodeb] (nodecardb) {$\cardof{B}$};
		\draw[->] (nodea)--(nodeb);
		\draw[<->] (nodea)--(nodecarda);
		\draw[<->] (nodeb)--(nodecardb);
		\draw[->, dashed] (nodecarda)--(nodecardb);
		\end{tikzpicture}
	\end{center}
	The double-headed arrows indicate !!{bijection}s, whose existence is guaranteed by \olref{lem:CardinalsExist}. In assuming that $\cardle{A}{B}$, there is some !!a{injection} to $A\to B$. Now, chasing the arrows around from $\cardof{A}$ to $A$ to $B$ to $\cardof{B}$, we obtain !!a{injection} $\cardof{A} \to \cardof{B}$ (the dashed arrow). 
\end{proof}\noindent
We can also use \olref{lem:CardinalsBehaveRight} to re-prove Schr\"{o}der--Bernstein. This is the claim that if $\cardle{A}{B}$ and $\cardle{B}{A}$ then $\cardeq{A}{B}$. We stated this as \olref[sfr][siz][sb]{schroderbernstein}, but first proved it---with some effort---in \olref[sfr][cardinals][card-sb]{card-sb}. Now consider:
\begin{proof}[Re-proof]
	If $\cardle{A}{B}$ and $\cardle{B}{A}$, then $\cardof{A} \leq \cardof{B}$ and $\cardof{B} \leq \cardof{A}$ by \olref{lem:CardinalsBehaveRight}. So $\cardof{A} = \cardof{B}$ and $\cardeq{A}{B}$ by Trichotomy and \olref{lem:CardinalsBehaveRight}.
\end{proof}\noindent
Whilst this is a very simple proof, it implicitly relies on both Replacement (to secure \olref[sfr][ordinals][ordtype]{thmOrdinalRepresentation}) and on Well-Ordering (to guarantee \olref{lem:CardinalsBehaveRight}). By contrast, the proof of \olref[sfr][cardinals][card-sb]{card-sb} was  much more self-standing (indeed, it can be carried out in $\Zminus$).

\end{document}