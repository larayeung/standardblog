\documentclass[../../../include/open-logic-section]{subfiles}

\begin{document}
		\olfileid{sfr}{ord-arithmetic}{ord-addition}
\olsection{Ordinal addition}\ollabel{ord-addition}
Suppose we want to add $\alpha$ and $\beta$. We can simply put a {copy} of $\beta$ immediately after a copy of $\alpha$. (We need to take \emph{copies}, since we know from \olref[sfr][ordinals][ord-basic]{ordinalsaresubsets} that either $\alpha \subseteq \beta$ or $\beta \subseteq \alpha$.) The intuitive effect of this is to run through an $\alpha$-sequence of stages, \emph{and then} to run through a $\beta$-sequence. The resulting sequence will be well-ordered; so by \olref[sfr][ordinals][ordtype]{thmOrdinalRepresentation} it is isomorphic to a (unique) ordinal. That ordinal can be regarded as the \emph{sum} of $\alpha$ and $\beta$. 

That is the intuitive idea behind ordinal addition. To define it rigorously, we start with the idea of taking \emph{copies} of sets. The idea here is to use arbitrary tags, $0$ and $1$, to keep track of which object came from where:
\begin{defn}\ollabel{defdissum}The \emph{disjoint sum} of $A$ and $B$ is $A \disjointsum B = (A\times \{0\}) \cup (B \times \{1\})$
\end{defn}\noindent
We next define an ordering on pairs of ordinals:
\begin{defn}
	For any ordinals $\alpha_1, \alpha_2, \beta_1, \beta_2$, say that:
	\begin{align*}
		\tuple{\alpha_1, \alpha_2} \rlexless \tuple{\beta_1, \beta_2}\text{ iff }& \text{either }\alpha_2 \in \beta_2\\
		& \text{or both }\alpha_2 = \beta_2\text{ and }\alpha_1 \in \beta_1
	\end{align*} 
\end{defn}\noindent 
This is a \emph{reverse lexicographic} ordering, since you order by the second element, then by the first. Now recall that we wanted to define $\alpha \ordplus \beta$ as the order type of a copy of $\alpha$ followed by a copy of $\beta$. To achieve that, we say:
\begin{defn}\ollabel{defordplus}
	For any ordinals $\alpha$, $\beta$, their sum is $\alpha \ordplus \beta = \ordtype{\alpha \disjointsum \beta, \rlexless}$.\footnote{This is a slight abuse of notation; strictly I should write ``$\Setabs{\tuple{x,y}\in \alpha\disjointsum\beta}{x \rlexless y}$'' in place of ``$\rlexless$''. But I trust no confusion can arise.}
\end{defn}\noindent
The following result, together with \olref[sfr][ordinals][ordtype]{thmOrdinalRepresentation}, confirms that our definition is well-formed:
\begin{lem}\ollabel{ordsumlessiswo} $\tuple{\alpha \disjointsum \beta, \rlexless}$ is a well-order, for any ordinals $\alpha$ and $\beta$.
\end{lem}
	\begin{proof}
	Obviously $\rlexless$ is connected on $\alpha \disjointsum \beta$. To show it is well-founded, fix a non-empty $X \subseteq \alpha \disjointsum \beta$, and let 
	$$X_0 = \Setabs{\tuple{a, b} \in X}{(\forall \tuple{x, y} \in X)b \leq y}$$
	now choose the element of $X_0$ with smallest first coordinate. 
\end{proof}\noindent 
So we have a lovely, explicit definition of ordinal addition. Here is an unsurprising fact (recall that  $1 = \{0\}$, by \olref[sfr][z][infinity-again]{defnomega}):
\begin{prop}
	$\alpha \ordplus  1 = \ordsucc{\alpha}$, for any ordinal $\alpha$
\end{prop}
\begin{proof}
	Consider the isomorphism $f$ from $\ordsucc{\alpha} = \alpha \cup \{\alpha\}$ to $\alpha\disjointsum1 = (\alpha \times \{0\}) \disjointsum (\{0\} \times \{1\})$ given by $f(\gamma) = \tuple{\gamma, 0}$ for $\gamma \in \alpha$, and $f(\alpha) = \tuple{0, 1}$.
\end{proof}\noindent
Moreover, it is easy to show that addition obeys certain recursive conditions:
\begin{lem}\ollabel{ordadditionrecursion}
	For any ordinals $\alpha, \beta$, we have:
	\begin{align*}
		\alpha\ordplus 0 &= \alpha\\
		\alpha \ordplus  (\beta\ordplus 1) &= (\alpha \ordplus  \beta) \ordplus  1\\
		\alpha  \ordplus  \beta &= \supstrict_{\delta < \beta}(\alpha \ordplus  \delta) && \text{if $\beta $ is a limit ordinal}
	\end{align*}
\end{lem}
\begin{proof}
	We check case-by-case; first:
		\begin{align*}
			\alpha \ordplus  0 &= \ordtype{(\alpha \times \{0\}) \disjointsum (0 \times \{1\}), \rlexless} \\
			&= \ordtype{(\alpha \times \{0\})\times\{0\}, \rlexless}\\
%			&= \ordtype{ \alpha, <} \\
			&= \alpha\\
	\alpha \ordplus  (\beta \ordplus  1) %&= \ordtype{(\alpha\times \{0\}) \cup (\ordtype{\beta \ordplus  1}\times \{1\}), \rlexless} \\
			&= \ordtype{(\alpha\times \{0\}) \cup (\ordsucc{\beta}\times \{1\}), \rlexless} \\
			&= \ordtype{(\alpha\times \{0\}) \cup (\beta \times \{1\}), \rlexless} \ordplus  1 \\
			&= (\alpha \ordplus  \beta) \ordplus  1
\end{align*}
Now let $\beta \neq \emptyset$ be a limit. If $\delta < \beta$ then also $\delta\ordplus 1 < \beta$, so $\alpha \ordplus  \delta$ is a proper initial segment of $\alpha \ordplus  \beta$. So $\alpha \ordplus  \beta$ is a strict upper bound on $X = \Setabs{\alpha \ordplus  \delta}{\delta < \beta}$. Moreover, if $\alpha \leq \gamma < \alpha \ordplus  \beta$, then clearly $\gamma = \alpha \ordplus  \delta$ for some $\delta < \beta$. So $\alpha \ordplus  \beta = \supstrict_{\delta< \beta}(\alpha\ordplus \delta)$. 
\end{proof}\noindent
But here is a striking fact. To define ordinal addition, we could \emph{instead} have simply used the Transfinite Recursion Theorem, and laid down the recursion equations, exactly as given in \olref{ordadditionrecursion} (though using ``$\ordsucc{\beta}$'' rather than ``$\beta \ordplus 1$'').

There are, then, two different ways to define operations on the ordinals. We can define them \emph{synthetically}, by explicitly constructing a well-ordered set and considering its order type. Or we can define them \emph{recursively}, just by laying down the recursion equations. Done correctly, though, the outcome is identical. For  \olref[sfr][ordinals][ordtype]{thmOrdinalRepresentation} guarantees that these recursion equations pin down \emph{unique} ordinals.

In many ways, ordinal arithmetic behaves just like addition of the natural numbers. For example, we can prove the following:
\begin{lem}\ollabel{ordinaladditionisnice}
	If $\alpha, \beta, \gamma$ are ordinals, then:
	\begin{enumerate}
		\item\ollabel{ordaddition1} if $\beta < \gamma$, then $\alpha \ordplus  \beta < \alpha \ordplus  \gamma$
		\item\ollabel{ordaddition2} if $\alpha \ordplus  \beta = \alpha\ordplus \gamma$, then $\beta = \gamma$
		\item\ollabel{ordaddition3}  $\alpha \ordplus  (\beta \ordplus  \gamma) = (\alpha \ordplus  \beta) \ordplus  \gamma$, i.e.\ addition is associative
		\item\ollabel{ordaddition4}  If $\alpha \leq \beta$, then $\alpha \ordplus  \gamma \leq \beta \ordplus  \gamma$
	\end{enumerate}
\end{lem}
\begin{proof}
	I prove \olref{ordaddition3}, leaving the rest as an exercise. The proof is by Simple Transfinite Induction on $\gamma$, using \olref{ordadditionrecursion}. When $\gamma = 0$:
			\begin{align*}
				(\alpha \ordplus  \beta) \ordplus  0 &= \alpha \ordplus  \beta  = \alpha \ordplus  (\beta \ordplus  0)
			\end{align*}		
			When $\gamma = \delta\ordplus 1$, suppose for induction that $(\alpha \ordplus  \beta) \ordplus  \delta = \alpha \ordplus  (\beta \ordplus  \delta)$; now:
			\begin{align*}
				(\alpha \ordplus  \beta) \ordplus  (\delta \ordplus  1) & = ((\alpha \ordplus  \beta) \ordplus  \delta)\ordplus 1\\
				& = (\alpha \ordplus  (\beta \ordplus  \delta)) \ordplus  1\\
				& = \alpha \ordplus  ((\beta \ordplus  \delta)\ordplus 1)\\
				& = \alpha \ordplus  (\beta \ordplus  (\delta\ordplus 1))
			\end{align*}	
			When $\gamma$ is a limit ordinal, suppose for induction that if $\delta \in \gamma$ then $(\alpha \ordplus  \beta) \ordplus  \delta = \alpha \ordplus  (\beta \ordplus  \delta)$; now:
			\begin{align*}
				(\alpha \ordplus  \beta) \ordplus  \gamma & = \supstrict_{\delta < \gamma}((\alpha \ordplus  \beta) \ordplus  \delta) \\
				&= \supstrict_{\delta < \gamma}(\alpha \ordplus  (\beta \ordplus  \delta))\\
					&= \alpha \ordplus  \supstrict_{\delta < \gamma}(\beta \ordplus  \delta)\\
				& = \alpha \ordplus  (\beta \ordplus  \gamma)
			\end{align*}
\end{proof}\noindent 
In these ways, ordinal addition should be very familiar. 
\begin{prob}
	Prove the remainder of \olref[sfr][ord-arithmetic][ord-addition]{ordinaladditionisnice}.
\end{prob}\noindent
But, there is a crucial way in which ordinal addition is \emph{not} like addition on the natural numbers.
\begin{prop}\ollabel{ordsumnotcommute}Ordinal addition is {not} commutative; $1 \ordplus  \omega = \omega < \omega \ordplus  1$.
	\begin{proof}
		Note that 
		$
			1 \ordplus  \omega = \supstrict_{n < \omega} (1 \ordplus n) = \omega \in \omega \cup \{\omega\} = \ordsucc{\omega} = \omega \ordplus  1$.
	\end{proof}
\end{prop}\noindent
Whilst this may initially come as a surprise, \emph{it shouldn't}. On the one hand, when you consider $1 \ordplus  \omega$, you are thinking about the order type you get by putting an extra element \emph{before} all the natural numbers. Reasoning as we did with Hilbert's Hotel in \olref[sfr][infinite][hilbert]{hilbert}, intuitively, this extra first element shouldn't make any difference to the overall order type. On the other hand, when you consider $\omega \ordplus  1$, you are thinking about the order type you get by putting an extra element \emph{after} all the natural numbers. And that's a radically different beast!

\olsection{Using ordinal addition}
Using addition on the ordinals, we can explicitly calculate the ranks of various sets, in the sense of \olref[sfr][spine][rank]{defnsetrank}:
\begin{lem}\ollabel{rankcomputation}
	If $\setrank{A} = \alpha$ and $\setrank{B} = \beta$, then:
	\begin{enumerate}
		\item\ollabel{exrankpow} $\setrank{\Pow{A}} = \alpha\ordplus 1$
		\item\ollabel{exrankpair} $\setrank{\{A, B\}} = \max(\alpha, \beta) \ordplus  1$
		\item\ollabel{exrankcup} $\setrank{A \cup B} = \max(\alpha, \beta)$
		\item\ollabel{exranktuple} $\setrank{\tuple{A,B}} = \max(\alpha, \beta) \ordplus  2$
		\item\ollabel{exranktimes} $\setrank{A \times B} \leq \max(\alpha, \beta) \ordplus  2$
		\item\ollabel{exrankunion} $\setrank{\bigcup A} = \alpha$ when  $\alpha$ is empty or a limit; $\setrank{\bigcup A} = \gamma$ when $\alpha = \gamma\ordplus 1$
	\end{enumerate}
\end{lem}
\begin{proof}
	Throughout, we invoke \olref[sfr][spine][rank]{ranksupstrict} repeatedly.
	
	\emph{\olref{exrankpow}.} If $x \subseteq A$ then $\setrank{x} \leq \setrank{A}$. So $\setrank{\Pow{A}} \leq \alpha \ordplus  1$. Since $A \in \Pow{A}$ in particular, $\setrank{\Pow{A}} = \alpha \ordplus  1$.
	
	\emph{\olref{exrankpair}.} By \olref[sfr][spine][rank]{ranksupstrict}
		
	\emph{\olref{exrankcup}.} By \olref[sfr][spine][rank]{ranksupstrict}.

	\emph{\olref{exranktuple}.} By \olref{exrankpair}, twice.
	
	\emph{\olref{exranktimes}.} Note that $A \times B \subseteq \Pow{\Pow{A \cup B}}$, and invoke \olref{exranktuple}. 
	
	\emph{\olref{exrankunion}.} If $\alpha = \gamma\ordplus 1$, there is some $c \in A$ with $\setrank{c} = \gamma$, and no !!{element} of $A$ has higher rank; so $\setrank{\bigcup A} = \gamma$. If $\alpha$ is a limit ordinal, then $A$ has !!{element}s with rank arbitrarily close to (but strictly less than) $\alpha$, so that $\bigcup A$ also has !!{element}s with rank arbitrarily close to (but strictly less than) $\alpha$, so that $\setrank{\bigcup A} = \alpha$.
\end{proof}\noindent
I leave it as an exercise to show why \olref{exranktimes} involves an \emph{in}equality.
\begin{prob}
	Produce sets $A$ and $B$ such that $\setrank{A \times B}= \max(\setrank(A), \setrank{B})$. Produce sets $A$ and $B$ such that $\setrank{A \times B}\max(\setrank(A), \setrank{B}) \ordplus  2$. Are any other ranks possible?
\end{prob}

We are also now in a position to prove that several reasonable notions of ``finite'' coincide, when considering ordinals:
\begin{lem}\ollabel{ordinfinitycharacter}For any ordinal $\alpha$, the following are equivalent:
	\begin{enumerate}
		\item\ollabel{ord:notinomega} $\alpha\notin \omega$, i.e.\ $\alpha$ is not a natural number
		\item\ollabel{ord:omegaplus} $\omega \leq \alpha$ 
		\item\ollabel{ord:oneplus} $1 \ordplus  \alpha = \alpha$ %\alpha \approx \alpha \ordplus 1$
		\item\ollabel{ord:plusone} $\alpha \approx \alpha\ordplus 1$, i.e.\ $\alpha$ and $\alpha\ordplus 1$ are equinumerous
		\item\ollabel{ord:infinite} $\alpha$ is Dedekind infinite	
	\end{enumerate}
\end{lem}
\begin{proof}
	\emph{\olref{ord:notinomega} $\Rightarrow$ \olref{ord:omegaplus}.} By Trichotomy. 
		
	\emph{\olref{ord:omegaplus} $\Rightarrow$ \olref{ord:oneplus}.} Fix $\alpha \geq \omega$. By Transfinite Induction, there is some least ordinal $\gamma$ (possibly $0$) such that there is a limit ordinal $\beta$ with $\alpha = \beta \ordplus \gamma$. Now:
	$$1 \ordplus \alpha =  1 \ordplus (\beta \ordplus \gamma) = (1 \ordplus \beta) \ordplus \gamma =  \supstrict_{1 \ordplus  \delta} (\delta < \beta) \ordplus  \gamma = \beta \ordplus  \gamma = \alpha$$
	
	\emph{\olref{ord:oneplus} $\Rightarrow$ \olref{ord:plusone}.} There is clearly !!a{bijection} $f \colon (\alpha \disjointsum 1) \to (1 \disjointsum \alpha)$. If $1 \ordplus \alpha = \alpha$, there is an isomorphism $g \colon (1 \disjointsum \alpha) \to \alpha$. Now consider $\comp{f}{g}$.
	
	\emph{\olref{ord:plusone} $\Rightarrow$ \olref{ord:infinite}.} If $\alpha \approx \alpha \ordplus 1$, there is !!a{bijection} $f \colon (\alpha \disjointsum 1) \to \alpha$. Define $g(\gamma) = f(\gamma, 0)$ for each $\gamma < \alpha$; this !!{injection} witnesses that $\alpha$ is Dedekind infinite, since $f(0,1) \in \alpha \setminus \ran{g}$. 
	
	\emph{\olref{ord:infinite} $\Rightarrow$ \olref{ord:notinomega}.} This is \olref[sfr][z][infinity-again]{naturalnumbersarentinfinite}.
\end{proof}

\end{document}