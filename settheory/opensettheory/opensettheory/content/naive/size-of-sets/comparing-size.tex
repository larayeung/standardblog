% Part:sets-functions-relations
% Chapter: sets
% Section: comparing-sizes

\documentclass[../../../include/open-logic-section]{subfiles}

\begin{document}

\olfileid{sfr}{siz}{car}

\olsection{Sets of different sizes, and Cantor's Theorem}

\begin{explain}
	We have offered a precise statement of the idea that two sets have the same
size. We can
also offer a precise statement of the idea that one set is smaller than another. 
Our definition of
``is smaller than (or equinumerous)'' will require, instead of a
!!{bijection} between the sets, !!a{injection} from the first
set to the second. If such a function exists, the size of the first
set is less than or equal to the size of the second. Intuitively,
!!a{injection} from one set to another guarantees that the range of
the function has at least as many !!{element}s as the domain, since no two
!!{element}s of the domain map to the same !!{element} of the range.
\end{explain}
\begin{defn}
$A$ is \emph{no larger than}~$B$, written $\cardle{A}{B}$, iff there
  is !!a{injection} $f \colon A \to B$.
\end{defn}\noindent
It is clear that this is a reflexive and transitive relation, but that it is not symmetric (this is left as an exercise). We can also introduce a notion, which states that one set is (strictly) smaller than another. 
\begin{defn}
$A$ is \emph{smaller than}~$B$, written $\cardless{A}{B}$, iff
  there is !!a{injection}~$f\colon A \to B$ but no
  !!{bijection}~$g\colon A \to B$, i.e.\ $\cardle{A}{B}$ and $\cardneq{A}{B}$.
\end{defn}\noindent
It is clear that this is relation is anti-reflexive, anti-symmetric, and transitive. (This is left as an exercise.) Using this notation, we can say that a set $A$ is !!{enumerable} iff $\cardle{A}{\Nat}$, and that $A$ is !!{nonenumerable} iff $\cardless{\Nat}{A}$. This allows us to restate \olref[sfr][siz][nen]{thm-nonenum-pownat} as the observation that $\cardless{\Nat}{\Pow{\Nat}}$. In fact, \citet{Cantor1892} proved that this last point is \emph{perfectly general}:
\begin{thm}[Cantor]\ollabel{thm:cantor}
$\cardless{A}{\Pow{A}}$, for any set $A$
\end{thm}
\begin{proof}
	The map $f(x) = \{x\}$ is !!a{injection} $f \colon A \to \Pow{A}$, so that $\cardle{A}{\Pow{A}}$. It remains to show that $\cardneq{A}{\Pow{A}}$. So, for reductio, suppose $A \approx \Pow{A}$, i.e.\ there is some bijection $g \colon A \to \Pow{A}$. Now consider:
	\begin{align*}
		D &:= \Setabs{x \in A}{x \notin g(x)}
	\end{align*}
	Note that $D \subseteq A$, so that $D \in \Pow{A}$. Since $g$ is !!a{bijection}, there is some $c \in A$ such that $g(c) = D$. But now reason thus:
	\begin{align*}
		c \in g(c) \text{ iff }c \in D \text{ iff }c \notin g(c)
	\end{align*}
	This is a contradiction; so $\cardneq{A}{\Pow{A}}$.
%The function~$f \colon X \to \Pow{X}$ that maps any $x \in X$ to its
%singleton~$\{x\}$ is !!{injective}, since if $x \neq y$ then also $f(x) =
%\{x\} \neq \{y\} = f(y)$.
%
%There cannot be !!a{surjective} function~$g\colon X \to \Pow{X}$, let
%alone a !!{bijective} one. For suppose that $g\colon X \to \Pow{X}$.
%Since $g$ is total, every $x \in X$ is mapped to a subset $g(x)
%\subseteq X$. We show that $g$ cannot be surjective. To do this, we
%define a subset~$Y \subseteq X$ which by definition cannot be in the
%range of~$g$. Let
%\[
%\overline{Y} = \Setabs{x \in X}{x \notin g(x)}.
%\]
%Since $g(x)$ is defined for all $x \in X$, $\overline{Y}$ is clearly a
%well-defined subset of~$X$.  But, it cannot be in the range
%of~$g$. Let $x \in X$ be arbitrary, we show that $\overline{Y} \neq
%g(x)$.  If $x \in g(x)$, then it does not satisfy $x \notin g(x)$, and
%so by the definition of~$\overline{Y}$, we have $x \notin
%\overline{Y}$.  If $x \in \overline{Y}$, it must satisfy the defining
%property of~$\overline{Y}$, i.e., $x \notin g(x)$.  Since $x$ was
%arbitrary this shows that for each $x \in X$, $x \in g(x)$ iff $x
%\notin \overline{Y}$, and so $g(x) \neq \overline{Y}$.  So
%$\overline{Y}$ cannot be in the range of~$g$, contradicting the
%assumption that~$g$ is surjective.
\end{proof}

\begin{explain}
  It's instructive to compare the proof of \olref{thm:cantor} to that
  of \olref[nen]{thm-nonenum-pownat}. There we showed that for any
  list $N_1$, $N_2$, \dots, of subsets of~$\Nat$ we can construct a
  set~$D$ of numbers guaranteed not to be on the list. It
  was guaranteed not to be on the list because $n \in N_n$ iff $n \notin D$, for every $n \in
  \Nat$. We follow the same idea here,
  except the indices~$n$ are now !!{element}s of~$A$ rather than
  of~$\Nat$. The set $D$ is defined so that it is
  different from~$g(x)$ for each $x \in A$, because $x \in g(x)$ iff
  $x \notin D$. %Again, there is always !!a{element} of~$X$
%  which is !!a{element} of one of $g(x)$ and $D$ but not the other. 
  
  The proof is also worth comparing with the proof of Russell's Paradox, \olref[sfr][set][rus]{RussellsParadox}. Indeed, Cantor's Theorem was the inspiration for Russell's own paradox.
\end{explain}

%\begin{prob}
%  Show that there cannot be !!a{injection} $g\colon \Pow(X) \to
%  X$, for any set $X$. Hint: Suppose $g\colon \wp(X) \to X$ is
%  !!{injective}. Then for each $x \in X$ there is at most one $Y \subseteq
%  X$ such that $g(Y) = x$. Define a set $\overline{Y}$ such that for
%  every $x \in X$, $g(\overline{Y}) \neq x$.
%\end{prob}

\end{document}
