% Part: sets-functions-relations
% Chapter: relations
% Section: special-properties

\documentclass[../../../include/open-logic-section]{subfiles}

\begin{document}

\olfileid{sfr}{rel}{prp}
\olsection{Special Properties of Relations}

\begin{intro}
Some kinds of relations turn out to be so common that they have been
given special names.  For instance, $\le$ and $\subseteq$ both relate
their respective domains (say, $\Nat$ in the case of $\le$ and
$\Pow{X}$ in the case of $\subseteq$) in similar ways.  To get at
exactly how these relations are similar, and how they differ, we
categorize them according to some special properties that relations
can have.  It turns out that (combinations of) some of these special
properties are especially important: orders and equivalence relations.
\end{intro}

\begin{defn}[Reflexivity]
A relation $R \subseteq X^2$ is \emph{reflexive} iff, for every $x \in
X$, $Rxx$.
\end{defn}

\begin{defn}[Transitivity]
A relation $R \subseteq X^2$ is \emph{transitive} iff, whenever $Rxy$
and $Ryz$, then also $Rxz$.
\end{defn}

\begin{defn}[Symmetry]
A relation~$R \subseteq X^2$ is \emph{symmetric} iff, whenever
$Rxy$, then also~$Ryx$.
\end{defn}

\begin{defn}[Anti-symmetry]
A relation~$R \subseteq X^2$ is \emph{anti-symmetric} iff, whenever both
$Rxy$ and $Ryx$, then $x=y$ (or, in other words: if $x\neq y$ then
either $\lnot Rxy$ or $\lnot Ryx$).
\end{defn}

\begin{explain}
In a symmetric relation, $Rxy$ and $Ryx$ always hold together, or
neither holds.  In an anti-symmetric relation, the only way for $Rxy$
and $Ryx$ to hold together is if $x = y$.  Note that this does not
\emph{require} that $Rxy$ and $Ryx$ holds when $x = y$, only that it
isn't ruled out.  So an anti-symmetric relation can be reflexive, but
it is not the case that every anti-symmetric relation is
reflexive.  Also note that being anti-symmetric and merely not being
symmetric are different conditions.  In fact, a relation can be both
symmetric and anti-symmetric at the same time (e.g., the identity
relation is).
\end{explain}

\begin{defn}[Connectivity]
A relation $R \subseteq X^2$ is \emph{connected} if for all $x,y\in
X$, if $x \neq y$, then either $Rxy$ or~$Ryx$.
\end{defn}

\begin{defn}[Partial order]
A relation~$R \subseteq X^2$ that is reflexive, transitive, and
anti-symmetric is called a \emph{partial order}. 
\end{defn}

\begin{defn}[Linear order]
A partial order that is also connected is called a \emph{linear order}.
\end{defn}

\begin{defn}[Equivalence relation]
A relation $R \subseteq X^2$ that is reflexive, symmetric, and
transitive is called an \emph{equivalence relation}.
\end{defn}\noindent
\begin{prob}
	Give examples of relations that are (a) reflexive and symmetric but
	not transitive, (b) reflexive and anti-symmetric, (c) anti-symmetric,
	transitive, but not reflexive, and (d) reflexive, symmetric, and
	transitive.  Do not use relations on numbers or sets.
\end{prob} 
Equivalence relations give rise to the notion of an \emph{equivalence class}. An equivalence relation ``chunks up'' the domain into different partitions; within each partition, all the objects are related to one another; and no objects from different partitions relate to one another. Sometimes, it's helpful just to talk about these partitions \emph{directly}. To that end, we introduce a definition:
\begin{defn}\ollabel{defequivalenceclass}
	Let $R \subseteq X^2$ be an equivalence relation. For each $a \in X$, the \emph{equivalence class} of $a$ in $X$ is the set $\equivrep{a}{R} = \Setabs{x \in X}{Rax}$. The \emph{quotient} of $X$ under $R$, is $\equivclass{X}{R} = \Setabs{\equivrep{a}{R}}{a \in X}$, i.e.\ the set of these equivalence classes.
\end{defn}\noindent
This next result vindicates the definition of an equivalence class, in proving that the equivalence classes are indeed the partitions of $X$:
\begin{prop}
 	If $R \subseteq X^2$ is an equivalence relation, then $Rab$ iff $\equivrep{a}{R} = \equivrep{b}{R}$.
 \end{prop}
\begin{proof}
	\emph{Left to right.} Suppose $Rab$, and let $c \in \equivrep{a}{R}$. By definition, then, $Rac$. Since $R$ is an equivalence relation, $Rbc$. (Spelling this out: as $Rab$ and $R$ is symmetric we have $Rba$, and as $Rac$ and $R$ is transitive we have $Rbc$.) So $c \in \equivrep{b}{R}$. Generalising, $\equivrep{a}{R} \subseteq \equivrep{b}{R}$. But exactly similarly, $\equivrep{b}{R} \subseteq \equivrep{a}{R}$. So $\equivrep{a}{R} = \equivrep{b}{R}$, by Extensionality.
	
	\emph{Right to left.} Suppose $\equivrep{a}{R} = \equivrep{b}{R}$. Since $R$ is symmetric, $Rbb$, so $b \in \equivrep{b}{R}$. So $b \in \equivrep{a}{R}$. So $Rab$.
\end{proof}
\begin{ex}
	 A nice example of equivalence relations comes from modular arithmetic. For any $a, b, n \in \Nat$, say that $a \equiv_n b$ iff dividing $a$ by $n$ gives remainder $b$. (Somewhat more symbolically: $a \equiv_n b$ iff $(\exists k \in \Nat)a -b = kn$.) Now, $\equiv_n$ is an equivalence relation, for any $n$. And there are exactly $n$ distinct equivalence classes generated by $\equiv_n$; that is, $\equivclass{\Nat}{\equiv_n}$ has $n$ !!{element}s. These are: the set of numbers divisible by $n$ without remainder, i.e.\ $\equivrep{0}{\equiv_n}$; the set of numbers divisible by $n$ with remainder $1$, i.e.\ $\equivrep{1}{\equiv_n}$; \ldots; and the set of numbers divisible by $n$ with remainder $n-1$, i.e.\ $\equivrep{n-1}{\equiv_n}$. 
\end{ex}
\begin{prob}
	Show that $\equiv_n$ is an equivalence relation, for any $n \in \Nat$, and that $\equivclass{\Nat}{\mod_n}$ has exactly $n$ members.
\end{prob}


\end{document}