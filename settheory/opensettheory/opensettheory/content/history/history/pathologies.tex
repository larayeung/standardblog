\documentclass[../../../include/open-logic-section]{subfiles}

\begin{document}
	\olfileid{sfr}{history}{pathology}
	\olsection{Pathologies}\ollabel{pathology}
However, the definition of a \emph{limit} turned out to allow for some rather ``pathological'' constructions. 

Around the 1830s, Bolzano discovered a function which was \emph{continuous everywhere}, but \emph{differentiable nowhere}. (Unfortunately, Bolzano never published this; the idea was first encountered by mathematicians in 1872, thanks to Weierstrass's independent discovery of the same idea.)\footnote{The history is documented in extremely thorough footnotes to the Wikipedia article on \href{http://en.wikipedia.org/wiki/Weierstrass_function}{the Weierstrass function}.} This was, to say the least, rather surprising. It is easy to find functions, such as $|x|$, which are continuous everywhere but not differentiable at a particular point. But a function which is continuous everywhere but differentiable \emph{nowhere} is a very different beast. Consider, for a moment, how you might try to draw such a function. To ensure it is continuous, you must be able to draw it without ever removing your pen from the page; but to ensure it is differentiable nowhere, you would have to abruptly change the direction of your pen, constantly.

Further ``pathologies'' followed. In January 5 1874, Cantor wrote a letter to Dedekind, posing the problem:
\begin{quote}
	Can a surface (say a square including its boundary) be one-to-one correlated to a line (say a straight line including its endpoints) so that to every point of the surface there corresponds a point of the line, and conversely to every point of the line there corresponds a point of the surface?
	
	It still seems to me at the moment that the answer to this question is very difficult---although here too one is so impelled to say \emph{no} that one would like to hold the proof to be almost superfluous. [Quoted in \citeauthor{Gouvea2011} \citeyear{Gouvea2011}]
\end{quote}
But, in 1877, Cantor proved that he had been wrong. In fact, a line and a square have exactly the same number of points. He wrote on 29 June 1877 to Dedekind ``\emph{je le vois, mais je ne le crois pas}''; that is, ``I see it, but I don't believe it''. In the ``received history'' of mathematics, this is often taken to indicate just how \emph{literally incredible} these new results were to the mathematicians of the time. (The correspondence is presented in \citeauthor{Gouvea2011} \citeyear{Gouvea2011}, and I return to it in \olref[sfr][history][mythology]{mythology}. I will outline Cantor's proof in \olref[sfr][cardinals][cantorplane]{cantorplane}.) 

Inspired by Cantor's result, Peano started to consider whether it might be possible to map a line \emph{smoothly} onto a plane. This would be a \emph{curve which fills space}. In \citeyear{Peano1890}, Peano constructed just such a curve. This is truly counter-intuitive: Euclid had defined a line as ``breadthless length'' (Book I, Definition 2), but Peano had shown that, by curling up a line appropriately, its length can be turned into breadth. In \citeyear{Hilbert1891}, Hilbert described a slightly more intuitive space-filling curve, together with some pictures illustrating it. The curve is constructed in sequence, and here are the first six stages of the construction:\footnote{For credit on these pictures: Aristid Lindenmayer developed the Lindenmayer System for describing models of growth, which has here been implemented in the lindenmayersystems library for TikZ. I was also inspired by the Wikipedia entry on \href{http://en.wikipedia.org/wiki/Space-filling_curve}{space-filling curves}.}
\begin{center}
\begin{tikzpicture}
\begin{scope}[xshift=20pt, yshift=20pt]
	\draw [darkred, l-system={Hilbert curve, step=40pt, angle=90, axiom=L, order=1}] lindenmayer system;
\end{scope}
\begin{scope}[xshift=110pt, yshift=10pt]
	\draw [darkred, l-system={Hilbert curve, step=20pt, angle=90, axiom=L, order=2}] lindenmayer system;
\end{scope}
\begin{scope}[xshift=205pt, yshift=5pt]
	\draw [darkred, l-system={Hilbert curve, step=10pt, angle=90, axiom=L, order=3}] lindenmayer system;
\end{scope}
\begin{scope}[xshift=2.5pt, yshift=-97.5pt]
	\draw [darkred, l-system={Hilbert curve, step=5pt, angle=90, axiom=L, order=4}] lindenmayer system;
\end{scope}
\begin{scope}[xshift=101.25pt, yshift=-98.75pt]
	\draw [darkred, l-system={Hilbert curve, step=2.5pt, angle=90, axiom=L, order=5}] lindenmayer system;
\end{scope}
\begin{scope}[xshift=200.625pt, yshift=-99.375pt]
	\draw [darkred, l-system={Hilbert curve, step=1.25pt, angle=90, axiom=L, order=6}] lindenmayer system;
\end{scope}
\draw[gray] (0pt,0pt) rectangle (80pt, 80pt);
\draw[gray] (100pt,0pt) rectangle (180pt, 80pt);
\draw[gray] (200pt,0pt) rectangle (280pt, 80pt);
\draw[gray] (0pt,-100pt) rectangle (80pt, -20pt);
\draw[gray] (100pt,-100pt) rectangle (180pt, -20pt);
\draw[gray] (200pt,-100pt) rectangle (280pt, -20pt);
\end{tikzpicture}  
\end{center}
In the limit---a notion which had, by now, received rigorous definition---the entire square is filled in solid red. And, in passing, Hilbert's curve is continuous everywhere but differentiable nowhere; intuitively because, in the infinite limit, the function abruptly changes direction at every moment. (I will outline Hilbert's construction in more detail in \olref[sfr][history][hilbertcurve]{hilbertcurve}.)

For better or worse, these ``pathological'' geometric constructions were treated as a reason to doubt appeals to geometric intuition. They became something approaching \emph{propaganda} for a new way of doing mathematics, which would culminate in set theory. In the later myth-building of the subject, it was repeated, often, that these results were both perfectly rigorous and perfectly shocking. They therefore served a dual purpose: as a warning against relying upon geometric intuition, and as a demonstration of the fertility of new ways of thinking. 

\end{document}