\documentclass[../../../include/open-logic-section]{subfiles}

\begin{document}
	\olfileid{sfr}{ordinals}{ord-wo}
	\olsection{Well orderings}\ollabel{ord-wo}
The fundamental notion is as follows:
\begin{defn}
	The relation $<$ \emph{well-orders} $A$ iff it meets these two conditions:
	\begin{enumerate}
		\item $<$ is connected, i.e.\ for all $a, b \in A$, either $a < b$ or $a = b$ or $b < a$ 
		\item every non-empty subset of $A$ has a $<$-minimal !!{element}, i.e., if $\emptyset \neq X \subseteq A$ then $(\exists m \in X)(\forall z \in X)z \nless m$
	\end{enumerate}
\end{defn}\noindent
It is easy to see that three examples we just considered were indeed well-ordering relations. 
\begin{prob}
	\Olref[sfr][ordinals][ord-idea]{ord-idea} presented three example orderings on the natural numbers. Check that each is a well-ordering.
\end{prob}\noindent 
Here are some elementary but extremely important observations concerning well-ordering.
\begin{prop}\ollabel{wo:strictorder}
	If $<$ well-orders $A$, then every non-empty subset of $A$ has a $<$-least member, and $<$ is irreflexive, asymmetric and transitive.
\end{prop}
\begin{proof}
	If $X$ is a non-empty subset of $A$, it has a $<$-minimal !!{element} $m$, i.e.\ $(\forall z \in X)z \nless m$. Since $<$ is connected, $(\forall z \in X)m \leq z$. So $m$ is $<$-least.
	
	For irreflexivity, fix $a \in A$; since $\{a\}$ has a $<$-least !!{element}, $a \nless a$. For transitivity, if $a < b < c$, then since $\{a, b, c\}$ has a $<$-least !!{element}, $a < c$. Asymmetry follows from irreflexivity and transitivity
\end{proof}
\begin{prop}\ollabel{propwoinduction}
	If $<$ well-orders $A$, then for any formula $\phi(x)$:\footnote{which may have parameters} 
	$$\text{if }(\forall a \in A)((\forall b < a)\phi(b) \lonlyif \phi(a))\text{, then }(\forall a \in A)\phi(a)$$
\end{prop}
\begin{proof}
	I will prove the contrapositive. Suppose $\lnot(\forall a \in A)\phi(a)$, i.e.\ that $X = \Setabs{x \in A}{\lnot\phi(x)} \neq \emptyset$. Then $X$ has an $<$-minimal !!{element}, $a$. So $(\forall b < a)\phi(b)$ but $\lnot \phi(a)$. 
\end{proof}\noindent
This last property should remind you of the principle of strong induction on the naturals, i.e.: if $(\forall n \in \omega)((\forall m < n)\phi(m) \lonlyif \phi(n))$, then $(\forall n \in \omega)\phi(n)$. And this property makes well-ordering into a very \emph{robust} notion. 

\end{document}